\documentclass[a4paper,11pt]{article}

\usepackage{graphicx}
\usepackage[ansinew]{inputenc}
\usepackage[spanish]{babel}
\usepackage{multirow}
\usepackage{fancyhdr}
\usepackage{url}
\usepackage{listings}
\usepackage{color}

\title{66:20 Organizaci�n de Computadoras\\
        Principios de conjuntos de instrucciones / Assembly MIPS}
\author{}
\sloppy
\date{}

\begin{document}
\lstset{%
  basicstyle=\small\ttfamily,
  breaklines=true,
  tabsize=2,
  language=C,
  extendedchars=true
}

\lstdefinestyle{6620C}{
  identifierstyle=\color{blue}
%  stringstyle=\color{purple}
}

\maketitle
\begin{center}
Complementar esta guia de ejercicios con los ejercicios 2.1, 2.2, 2.3, 2.4, 2.6, 2.7, 2.8, 2.9, 2.10, 2.11, 2.12, 2.13*, 2.14, 2.18, 2.19  de la edici�n 3 del CAAQA.
\end{center}

\hrule

\thispagestyle{empty} 

\begin{enumerate}
  \item %Ejercicio 1
	  Pasar el siguiente c�digo C a MIPS.
		\begin{small}
			\begin{verbatim}
				Asumir que las seis variables (f,g,h,i,j,k) corresponden a seis registros 
				($s0,$s1,$s2,$s3,$s4,$s5), y existe una variable $t2=4.
				 
				switch (k) {
					case 0:
				             f = i + j; break; 
					case 1:
				             f = g + h; break;
					case 2:
				             f = g - h; break; 
					case 3:
				             f = i - j; break; 
				}
			\end{verbatim}
		\end{small}
		\textit{Pista: usar la variable k para indexar una tabla de saltos, y luego 
		saltar a dichos valores.
		Primero verificar que el valor k se corresponda con una de las opciones posibles
		(0 \(<=\) k \(<=\) 3 ). Si no, salir.}
		
		
		%RESPUESTA:
		
%		.data
%		TablaSaltos:
%		.word L0, L1, L2, L3
%		.text
%	
%		slt $t3, $s5, $zero 		# Probar k < 0
%		bne $t3, $zero, Salida 		# Si es, Salida
%		slt $t3, $s5, $t2 		# Probar k < 4
%		beq $t3, $zero, Salida 		# Si no, Salida
%		sll $t1, $s5, 2 		# $t1 = 4*k
%		li $t4, TablaSaltos		# Carga en $t4 el valor TablaSaltos
%		add $t1, $t1, $t4 		# $t1 = &TablaSaltos[k]
%		lw $t0, 0($t1) 			# $t0 = TablaSaltos[k]
%		jr $t0 				# jump register
%	
%	L0: 	add $s0, $s3, $s4 		# k == 0
%		j Salida 			# break
%	L1: 	add $s0, $s1, $s2	 	# k == 1
%		j Salida 			# break
%	L2: 	sub $s0, $s1, $s2 		# k == 2
%		j Salida 			# break
%	L3: 	sub $s0, $s3, $s4	 	# k == 3
%		j Salida 			# break
		
		
		
  \item %Ejercicio 2
  
  [2.12 de H.P AQA]\\
  
		  \begin{small}
			\begin{verbatim}                                   
			Instruction     gap    gcc    gzip   mcf    perl   I. average
			load           44.7%   35.5%  31.8%  33.2%  41.6%   37%
			store          10.3%   13.2%  5.1%   4.3%   16.2%   10%
			add             7.7%   11.2%  16.8%  7.2%   5.5%    10%
			sub             1.7%   2.2%   5.1%   3.7%   2.5%    3%
			mul             1.4%   0.1%                         0%
			compare         2.8%   6.1%   6.6%   6.3%   3.8%    5%
			cond branch     9.3%   12.1%  11.0%  17.5%  10.9%   12%
			cond move       0.4%   0.6%   1.1%   0.1%   1.9%    1%
			jump            0.8%   0.7%   0.8%   0.7%   1.7%    1%
			call            1.6%   0.6%   0.4%   3.2%   1.1%    1%
			return          1.6%   0.6%   0.4%   3.2%   1.1%    1%
			shift           3.8%   1.1%   2.1%   1.1%   0.5%    2%
			and             4.3%   4.6%   9.4%   0.2%   1.2%    4%
			or              7.9%   8.5%   4.8%   17.6%  8.7%    9%
			xor             1.8%   2.1%   4.4%   1.5%   2.8%    3%
			other logical   0.1%   0.4%   0.1%   0.1%   0.3%    0%
			load FP                                             0%
			store FP                                            0%
			add FP                                              0%
			sub FP                                              0%
			mul FP                                              0%
			div FP                                              0%
			mov reg-reg FP                                      0%
			compare FP                                          0%
			cond mov FP                                         0%
			other FP                                            0%
			\end{verbatim}
			\end{small}
		
			Considerar el agregado de un nuevo modo de acceso a MIPS.
			El nuevo modo suma dos registros y un valor de offset de 11 bits con signo para obtener
			la direcci�n efectiva. Utilizar el porcentaje de instrucciones indicado en la tabla anterior.
			\vspace{3mm}

			El compilador pasa de las instrucciones: 
			\begin{verbatim}
			    add R1, R1, R2
			    lw  Rd, 100(R1) #(o store)
			\end{verbatim}
			\vspace{1.5mm}

			A utilizar:
			\begin{verbatim}
			    lw	Rd, 100(R1,R2)
			\end{verbatim}
			
			\subitem a.	Asumir que el nuevo modo de acceso es usado por el 10\% de los loads y stores.
			        �Cual es el porcentaje de Ic nuevo comparado con la tasa original?
			
			\subitem b.	Si el nuevo modo de direccionamiento aumenta en 5\% el tiempo de clock, �Cu�l
			       computadora ser� m�s r�pida y por cuanto?
			       
			       
  \pagebreak
			 %RESPUESTA
%			 
%			 a)
%
%				load/store = 47%
%				add = 10%
%				La nueva instrucci�n se usa para el 10% de los load/store, con lo cual:
%			
%				load/store|new = 0.1 x 0.47 x 100 = 4.7%
%			
%				El conteo total de load/store no cambia, porque cada una de estas
%				instrucciones reemplaza a 1 load/store. Pero desaparecen los add's
%				asociados a estos load/store. La cantidad de add's que desaparecen son
%				justamente IC|new_ins (cada vez que reemplazo el par de instrucciones
%				add+load/store, queda una sola instruccion load/store con el add
%				incluido).
%			
%				Del porcentaje total de instrucciones, seria:
%			
%				IC_new = IC_old - IC|new_ins = 1 - 0.047 * 100 = 95,3%
%			
%				La cantidad de estas instrucciones es IC|new_ins = IC_old x 0.047
%				
%			
%				b) El Tck aumenta en un 5%. Esto puede deberse a que el computo de la
%				direccion efectiva para las nuevas instrucciones toma mas tiempo, por
%				haber mas operandos involucrados.
%				
%				Tck_new = Tck_old x 1.05
%			
%				T = IC x CPI x Tck
%			
%				T_new = (IC_old x 0.953) x CPI_new x (Tck_old x 1.05)
%			                
%			             (es Tn/Tv por que se que me da peor)
%			  	Speedup = T_new / T_old = (IC_old x 0.953 x CPI_new x Tck_old x 1.05) /
%				                          (IC_old x CPI_old x Tck_old)
%							= 1.00065 x CPI_new / CPI_old
%				
%				Si CPI no cambia, el conteo de instrucciones se reduce, pero igual
%				aumenta el tiempo de ejecucion. Sin embargo es de esperarse que cambie
%				el CPI, puesto que para un programa dado, cambia la secuencia de
%				instrucciones que se procesan el pipeline, con lo cual en un caso real,
%				el resultado es incierto con estos datos.
%						       
     
  \item %Ejercicio 4  
  Codificar en Assembly MIPS y diagramar el stack de las siguientes funciones.
	\begin{small}
			\begin{verbatim}
			
			void proc(int i){
			  int j;
			  j = i+20; 
			} 

			int main(int argc, char** argv){
			    int i=10; 
			    proc(i); 
			    return 0; 
			} 
			
	\end{verbatim}
  \end{small}
  
  
  
  
  %RESPUESTA
%  .text
% .globl main 
%	main: 
%  subu $sp,$sp,48          # stack
%   sw $ra,40($sp)           # (almacena i y argumentos de main)
%   sw $fp,36($sp) 
%   sw $gp,32($sp) 
%   move $fp,$sp   
%   sw $a0,48($fp)            # el callee (main) almacena sus args 
%   sw $a1,52($fp)            # fuera del stack frame (pegados a el) 
%   li $v0,10                     # i = 10; 
%   sw $v0,24($fp)           #valor de retorno de proc(i)
%   lw $a0,24($fp)            # proc(i); 
%   la $t9,proc                 # cargo el identificador proc
%   jal $ra,$t9                  # llamada a la funcion
%   move $v0,$zero         # return 0; 
%   move $sp,$fp             # destruye stack frame 
%   lw $ra,40($sp) 
%   lw $fp,36($sp) 
%   addu $sp,$sp,48 
%   j $ra 
%
%	.globl proc 
%	
%	proc: 
%		subu $sp,$sp,24         # crea stack frame 
%		sw $fp,20($sp)           # necesita almacenar j 
%		sw $gp,16($sp) 
%		move $fp,$sp 
%		sw $a0,24($fp)         # callee (main) almacena argumento 
%		                               # fuera del stack frame 
%		lw $v0,24($fp)          # j = i + 20 
%		addu $v0,$v0,20 
%		sw $v0,8($fp)            # (j -> 8($fp)) 
%		move $sp,$fp            # destruye stack frame 
%		lw $fp,20($sp) 
%		addu $sp,$sp,24 
%		j $ra 

  
  
  \item %Ejercicio 5
  Codificar en Assembly MIPS y diagramar el stack de la siguiente funci�n.
		\begin{small}
			\begin{verbatim}
				unsigned int
				factorial (unsigned int n)
				{
				    if (n < 2)
				        return 1;
				    else 
				        return n*factorial(n-1);
				}
			\end{verbatim}
		\end{small}
  
  \item %Ejercicio 6
  Codificar en C  y Assembly MIPS una funci�n que reciba calcule la longitud de un C-string\\
  \begin{verbatim}
  size_t strlen(const char *s)
  \end{verbatim}

  
  \pagebreak
  \item 
  Codificar una funci�n que tome una linea de texto y retorne el entero correspondiente.
  	
  \lstinputlisting[style=6620C]{code/simple_atoi.c}
  \pagebreak
     
  \item %Ejercicio 4  
  Codificar en Assembly MIPS y diagramar el stack de las siguientes funciones.
	\begin{small}
			\begin{verbatim}
			
			void proc(int i){
			  int j;
			  j = i+20; 
			} 

			int main(int argc, char** argv){
			    int i=10; 
			    proc(i); 
			    return 0; 
			} 
			
	\end{verbatim}
  \end{small}
  
  
  \vfill
 

  \item%
  Codificar en C y Assembly MIPS una funci�n que transpone una matriz cuadrada de enteros de $n x n$

  \begin{verbatim}
	void mat_transp(int *m, size_t n);
  \end{verbatim}

  \item
  Codificar en C y Assembly MIPS una funci�n que indique si un n�mero es primo o no.

  \begin{verbatim}
	int prime(unsigned int n);
  \end{verbatim}
	 

   \vfill

  \item
  Codificar en Assembly MIPS una soluci�n al problema de Las Torres de Hanoi\footnote{\url{http://en.wikipedia.org/wiki/Tower_of_Hanoi}}

  \begin{verbatim}
    void
    hanoi(int n, int start, int finish, int extra)
    {
	    if (!n) return;

       hanoi(n-1, start, extra, finish);
       hanoi(n-1, extra, finish, start);
    }

  \end{verbatim}
  
 %RESPUESTA Ej 5
 
%		 #include <mips/regdef.h>
%		
%			.text
%			.ent factorial
%			.globl factorial
%			
%		factorial:	
%		
%			.frame 	$fp,40,ra
%		
%			subu 	sp,sp,40 	#Stack frame creation
%		
%			.cprestore 24		#Saved register area. 
%			sw	ra,32(sp)
%			sw	$fp,28(sp)	
%			move	$fp,sp
%		
%			sw	a0,40($fp)	#Argument Building Area
%			sw	a0,16($fp)	#Local and Temporary Area.
%						
%			lw	t0,16($fp)
%			li	t1,2
%			bltu	t0,t1,$ret1
%		
%		
%		$rec:	lw	t0,16($fp)	#n>=2
%			subu	t0,t0,1
%			sw	t0,16($fp)
%			addu	a0,t0,zero
%			jal	factorial	#factorial(n-1)
%			
%			lw	t0,40($fp)	
%			mul	v0,v0,t0	
%			b	$fin
%		
%		$ret1:	li	v0,1		#return 1
%		
%		$fin:	lw	ra,32(sp)
%			lw	$fp,28(sp)
%			addu	sp,sp,40
%		
%			jr	ra
%			.end 	factorial  
%

\end{enumerate}

\end{document}

