\documentclass{article}

%\usepackage[spanish]{babel}
%\usepackage[ansinew]{inputenc}
%% graficos
%\usepackage[pdftex]{graphicx}

\usepackage{graphicx}
\usepackage{tikz}
\usepackage[utf8]{inputenc}
\usepackage[spanish]{babel}
\usepackage{multirow}
\usepackage{fancyhdr}
\usepackage{url}
\usepackage{amsmath}

\newcommand {\SU}{\textit{Speed Up} }
\author{}
\title{Trabajo práctico 0: Ley de Amdahl}

\begin{document}
\date{}
\maketitle

\section{Introducción}

En la clase del 22/8 se presentaron las herramientas necesarias para compilar, depurar y ejecutar programas en un 
ambiente MIPS. Asímismo se describieron los fundamentos de la Ley de Amdahl\footnote{También expuesta en la clase
teórica del lunes 26/8}, y como puede ser usada para estimar y medir los alcances de una mejora a un proceso.

Este trabajo práctico consiste en estudiar la performance de una implementación de la Hormiga de Langton que 
denominaremos 'La hormiga artista'. 

\section{La hormiga artista}

Como parte de los recursos a utilizar en este trabajo práctico, se provee un número de implementaciones del problema.
Para todos los casos, se deben especificar los siguientes parámetros
\begin{itemize}
\item Las dimensiones de la grilla
\item La paleta de colores a utilizar
\item Un conjunto de reglas para realizar las rotaciones
\item La cantidad de movimientos a realizar
\end{itemize}

A su vez, se pueden especificar las siguientes opciones de compilación para generar distintas versiones
\begin{itemize}
\item \texttt{USE\_COL\_MAJOR}
Al momento de imprimir el estado de la grilla, la misma es recorrida 'por columnas'
\item \texttt{USE\_TABLES}
Las acciones de rotar y moverse hacia adelante se realizan en funciones independientes
\item \texttt{SANITY\_CHECK}
El programa se detiene en caso de que no se cumpla alguna condición
\end{itemize}

En particular, consideramos relevante las versiones expuestas en \ref{secc:impl}.

\subsection{Implementación}
\label{secc:impl}

Como primer paso, se pueden generar dos versiones distintas del programa, variando la implementación de la siguiente
función:
\begin{small}
\begin{verbatim}
void*
paint(void *artist_ant, void *gridfn, colour_fn next_colour, 
    rule_fn next_rotation, uint64_t iterations);
\end{verbatim}
\end{small}

Se presume que la versión base demora mas tiempo en ejecutar que la alternativa -compilada con la opción 
\texttt{USE\_TABLES}- puede completar la cantidad especificada de iteraciones en menos tiempo.  Para comprobar si esta 
afirmación es cierta, utilizaremos las herramientas \texttt{time} y \texttt{gprof}.

\subsection{Ejemplos}

Listamos las opciones utilizando el comando \texttt{--help}
\begin{verbatim}
./tp0_if --help
  ./tp0_if -g <grid_spec> -p <colour_spec> -r <rule_spec> -t <n>
  -g --grid: wxh
  -p --palette: Combination of RGBYNW
  -r --rules: Combination of LR
  -t --times: Iterations. If negative, it's complement will be used.
  -o --outfile: output file. Defaults to stdout.
  -h --help: Print this message and exit
  -v --verbose: Version number

Compile with -DSANITY_CHECK to enable runtime checks
Compile with -DUSE_TABLES to execute ant operations in separate functions
Compile with -DUSE_COL_MAJOR to traverse the grid in column-major order
\end{verbatim}

Medimos el tiempo en ejecutar diez mil operaciones en la menor grilla posible, y repetimos escalando la cantidad
de operaciones
\begin{verbatim}
time -p ./tp0_if -g 1x1 -p RGBW -r LLLL -t ((10 * 1000))  > /dev/null
real 0.10
user 0.06
sys 0.02
\end{verbatim}

\begin{verbatim}
time -p ./tp0_if -g 1x1 -p RGBW -r LLLL -t $((100 * 1000))  > /dev/null
real 0.18
user 0.15
sys 0.02
\end{verbatim}

\begin{verbatim}
time -p ./tp0_if -g 1x1 -p RGBW -r LLLL -t $((1000 * 1000))  > /dev/null
real 1.41
user 1.20
sys 0.01
\end{verbatim}

Repetimos, con una grilla significativamente mas grande
\begin{verbatim}
time ./tp0_if -g 1024x1024 -p RGBW -r LLLL -t $((10 * 1000))  > /dev/null

real	0m3.611s
user	0m3.072s
sys	0m0.032s
\end{verbatim}

\begin{verbatim}
time ./tp0_if -g 1024x1024 -p RGBW -r LLLL -t $((100 * 1000))  > /dev/null

real	0m3.178s
user	0m2.784s
sys	0m0.012s
\end{verbatim}

\begin{verbatim}
time ./tp0_if -g 1024x1024 -p RGBW -r LLLL -t $((1000 * 1000))  > /dev/null

real	0m3.414s
user	0m2.976s
sys	0m0.028s
\end{verbatim}

\section{Objetivos}
Tal como se menciona arriba, se disponen de dos implementaciones de la versión \texttt{paint}. 
El objetivo del trabajo práctico estudiar cuál es el máximo \SU posible al optimizar dicha función, para después  
luego contrastar esta hipótesis con las mediciones realizadas sobre la nueva implementación.

Se espera que los siguientes puntos estén incluidos en el análisis del problema
\begin{itemize}
\item ¿Cómo varía el tiempo de ejecución a medida que se cambian los parámetros del programa?
\item ¿Durante qué proporción de tiempo se puede aplicar la mejora?
\item ¿Cuál es el máximo \SU posible?
\item Análisis realizado con \texttt{gprof}
\item Mediciones relevantes realizadas con \texttt{time}
\item Comparaciones del tiempo de ejecución con distintos parámetros
\item Cálculo del \SU global y local
\end{itemize}

\pagebreak
\section{Recursos}\label{Recursos}

\begin{itemize}
\item Hormiga de Langton: \url{https://es.wikipedia.org/wiki/Hormiga_de_Langton}
\item Formato PPM: \url{http://netpbm.sourceforge.net/doc/ppm.html}
\item Imagemagick \url{https://imagemagick.org/index.php}
\item GProf guía rápida \url{https://web.eecs.umich.edu/~sugih/pointers/gprof_quick.html}
\item Manual gprof \url{https://linux.die.net/man/1/gprof}
\end{itemize}

\section{Condiciones de entrega}

\subsection{Fechas de entrega}
\begin{itemize}
\item 5/9/2019
\item 19/9/2019
\end{itemize}

\begin{itemize}
\item Carátula especificando los datos y contacto de los integrantes del grupo (dirección de correo electrónico, 
\textit{handle} de slack, ubicación del repositorio de código)
\item Decisiones relevantes sobre la implementación y resolución
\item Conclusiones con \underline{fundamentos reales}
\item Casos de prueba documentados
\item Código fuente, si aplica
\item Este enunciado
\end{itemize}
\end{document}
