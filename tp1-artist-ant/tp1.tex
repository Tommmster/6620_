\documentclass{article}

%\usepackage[spanish]{babel}
%\usepackage[ansinew]{inputenc}
%% graficos
%\usepackage[pdftex]{graphicx}

\usepackage{graphicx}
\usepackage{tikz}
\usepackage[utf8]{inputenc}
\usepackage[spanish]{babel}
\usepackage{multirow}
\usepackage{fancyhdr}
\usepackage{url}
\usepackage{amsmath}


\author{}
\title{Trabajo práctico 1: Conjunto de instrucciones MIPS}

\begin{document}
\date{}
\maketitle

\section{Introducción}

En la clase del jueves 22/8 presentamos las herramientas necesarias para poder compilar y ejecutar
programas escritos en C dentro de un ambiente MIPS

\section{Descripción}
El trabajo práctico consiste en implementar un problema denominado \textbf{"La hormiga artista"}. El mismo plantea un recorrido
sobre un espacio de dos dimensiones, donde el mismo está definido por reglas predefinidas.

Se dispone de un area bidimensional dividida en celdas cuyo tamaño es configurable, donde cada celda está pintada de un color\footnote{El color inicial de todo el espacio es el primero de la paleta especificada}. En la posición central se encuentra una
hormiga, orientada hacia el norte\footnote{En el formato de salida, hacia el encabezado del archivo}, que dispone de una paleta de colores. 
Al encontrarse sobre una celda, la hormiga cambia su orientación rotando hacia la izquierda (\texttt{L}) o derecha (\texttt{R}) según el 
color sobre el que se encuentre, pinta la celda utilizando el color de la paleta que sigue al último usado\footnote{El primer color a utilizar es el segundo 
color de la paleta} y se mueve a la celda que tiene delante. Esta secuencia de acciones se repite un número determinado de veces.
Finalmente se escribe el estado de la grilla por la salida estandar, en formato PPM.

La paleta de colores contiene una combinación de los siguientes colores: 
Rojo (\texttt{R}), Azul (\texttt{B}), Verde (\texttt{G}), Amarillo (\texttt{Y}), Blanco (\texttt{W}) y Negro (\texttt{N}).

\pagebreak
\section{El programa}

El programa debe tomar las siguientes opciones:
\begin{itemize}
\item Las dimensiones de la grilla
\item La paleta de colores
\item El conjunto de reglas para realizar las rotaciones
\item La cantidad de movimientos a realizar
\end{itemize}
\subsection{Implementación}
Como punto de partida se provee un esqueleto escrito en C en el cual se encuentran resueltos aspectos como la evaluación de parámetros, 
algunas estructuras y la impresión final en formato PPM (ver Sección \ref{Recursos}).  Sin embargo, el movimiento de la hormiga a través del
espacio bidimensional debe programarse y tiene la siguiente interfaz:

\begin{small}
\begin{verbatim}
void*
paint(void *artist_ant, void *gridfn, colour_fn next_colour, rule_fn next_rotation, uint32_t iterations)
\end{verbatim}
\end{small}

\pagebreak
\section{Ejemplos}

    \begin{verbatim}
    ./tp1 --help
      ./tp1 -g <dimensions> -p <colors> -r <rules> -t <n>
      -g --grid: wxh
      -p --palette: Combination of R|G|B|Y|N|W
      -r --rules: Combination of L|R
      -n --times: Iterations
      -h --help: Print this message and exit
      -v --verbose: Version number
    \end{verbatim}

    \begin{verbatim}
    $ tp1 --grid 3x3 --palette "R|G|B" --rules "L|R|L" --times 3
    P3
    3 3
    255

    255 0   0   255 0   0   255 0   0
    0   0   255 0   255 0   255 0   0 
    255 0   0   255 0   0   255 0   0
    \end{verbatim}
La imagen resultante sería:
\begin{center}
\begin{tikzpicture}
\draw[step=1cm,gray,thin] (-1,-1) grid (2,2);
\fill[red](-1,1) rectangle(0,2);
\fill[red](0,1) rectangle(1,2);
\fill[red](1,1) rectangle(2,2);
\fill[blue](-1,0) rectangle(1,1);
\fill[green](0,0) rectangle(1,1);
\fill[red](1,0) rectangle(2,1);
\fill[red](-1,-1) rectangle(0,0);
\fill[red](0,-1) rectangle(1,0);
\fill[red](1,-1) rectangle(2,0);
\end{tikzpicture}
\end{center}

    \begin{verbatim}
    $ tp1 --grid 3x3 --palette Y|N|B --rules L|R|L --times 2
    P3
    3 3
    255

    255 255 0   255 255 0   255 255 0
    0   0   255 0   0   0   255 255 0
    255 255 0   255 255 0   255 255 0
    \end{verbatim}
La imagen resultante sería:
\begin{center}
\begin{tikzpicture}
\draw[step=1cm,gray,thin] (-1,-1) grid (2,2);
\fill[yellow](-1,1) rectangle(0,2);
\fill[yellow](0,1) rectangle(1,2);
\fill[yellow](1,1) rectangle(2,2);
\fill[blue](-1,0) rectangle(1,1);
\fill[black](0,0) rectangle(1,1);
\fill[yellow](1,0) rectangle(2,1);
\fill[yellow](-1,-1) rectangle(0,0);
\fill[yellow](0,-1) rectangle(1,0);
\fill[yellow](1,-1) rectangle(2,0);
\end{tikzpicture}
\end{center}

    \begin{verbatim}
    $ tp1 --grid 2x2 --palette N|W --rules R|L --times 1
    P3
    2 2
    255

    0   0   0   0   0   0 
    0   0   0   255 255 255
    \end{verbatim}

La imagen resultante sería:
\begin{center}
\begin{tikzpicture}
\draw[step=1cm,gray,thin] (-1,-1) grid (1,1);
\fill[black](-1,0) rectangle(1,1);
\fill[black](0,0) rectangle(1,1);
\fill[black](-1,-1) rectangle(0,0);
\fill[yellow](0,-1) rectangle(1,0);
\end{tikzpicture}
\end{center}

\section{Extras}
\begin{itemize}
\item Mediciones de tiempo
\item Impacto sobre el \textit{cache}
\end{itemize}

\section{Recursos}\label{Recursos}

\begin{itemize}
\item Formato PPM: http://netpbm.sourceforge.net/doc/ppm.html
\end{itemize}

\section{Condiciones de entrega}

Las fechas de entrega para el trabajo práctico son el Jueves 25/4 y el jueves 9/5. La entrega debe incluir un informe
describiendo la resolución del trabajo práctico, que incluya:

\begin{itemize}
\item Carátula especificando los datos y contacto de los integrantes del grupo (dirección de correo electrónico, \textit{handle} de slack)
\item Instrucciones necesarias para compilar y ejecutar el programa
\item Decisiones relevantes sobre la implementación y resolución
\item Conclusiones con \underline{fundamentos reales}
\item Diagramas ilustrando la estructura del \textit{stack} de cada función relevante
\item Casos de prueba documentados
\item Código fuente
\item Este enunciado
\end{itemize}


\end{document}
