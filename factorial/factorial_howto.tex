\documentclass[a4paper,11pt]{article}

\usepackage{tikz}
\usetikzlibrary{decorations.pathreplacing}
\usetikzlibrary{positioning}
\usepackage{graphicx}
\usepackage[spanish]{babel}
\usepackage[utf8]{inputenc}
\usepackage{multirow}
\usepackage{fancyhdr}
\usepackage{url}
\usepackage{listings}
\usepackage{color}

\author{}
\sloppy
\date{}

\definecolor{mymauve}{rgb}{0.58,0,0.82}
\definecolor{mygreen}{rgb}{0,0.6,0}
\definecolor{mygray}{gray}{0.6}

\begin{document}
\lstset{%
  basicstyle=\small\ttfamily,
  breaklines=true,
  tabsize=2,
  language=C,
  extendedchars=true
}

\lstdefinestyle{6620C}{
  identifierstyle=\color{blue},
  commentstyle=\color{mygreen},
  keywordstyle=\bfseries\color{mymauve},
  frame=single,
  breaklines=true
%  stringstyle=\color{purple}
}

\section{Objetivos}

\section{Implementación}

\subsection{Código C}

\subsection{ABI}
Podemos categorizar esta función de la siguiente manera:

\begin{itemize}
\item Realiza llamadas a otras funciones (\textit{non-leaf})
\item Recibe al menos un parámetro
\end{itemize}

\subsubsection{SRA}
\subsubsection{ABA}


\begin{tikzpicture}[padding/.style={rectangle, fill=mygray},]


\draw (0,0) rectangle (4,1) node[midway] {gp};
\draw (0,1) rectangle (4,2) node[midway] {fp};
\draw (0,2) rectangle (4,3) node[midway] {ra};
\node[padding, f]  () {padding} ;

\draw [decorate,decoration={brace, mirror, amplitude=10pt},xshift=-4pt,yshift=0pt]
(5, 0 ) -- (5,4 ) node [black,midway, align=right] 
{\footnotesize $SRA$};


\draw (0, 0) rectangle (4,-1) ;
\draw (0, -1) rectangle (4,-2) ;
\draw (0, -2) rectangle (4,-3) ;
\draw (0, -3) rectangle (4,-4) ;
\draw [decorate,decoration={brace, amplitude=10pt},xshift=-4pt,yshift=0pt]
(5, 0 ) -- (5, -4 ) node [black, align=right] 
{\footnotesize $ABA$};
\end{tikzpicture}

\subsection{Implementación en Assembly mips}
Una posible implementación de esta función es la siguiente.

\lstinputlisting{factorial.S}

\section{Ejercicios}

\begin{enumerate}
\item A
\end{enumerate}
\end{document}
