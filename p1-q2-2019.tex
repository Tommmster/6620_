\documentclass{article}

\usepackage[utf8]{inputenc}
\usepackage[spanish]{babel}
\usepackage{graphicx}
\usepackage{fancyhdr}
\usepackage{lastpage}
\usepackage{amssymb}
\usepackage{epsfig}
\usepackage{siunitx}

\addtolength{\hoffset}{-2.0cm}
\addtolength{\voffset}{-1.5cm}
\setlength{\textwidth}{16cm}
\setlength{\textheight}{23.5cm}

\begin{document}
\pagestyle{fancyplain}

\lhead{Organización de Computadoras - 66:20\\
Parcial 24/10, $2^{do}$ cuatrimestre de 2019}
\rhead{página \thepage/\pageref{LastPage}}
\lfoot{{\footnotesize 2019/10/24}}
\cfoot{}
\rfoot{}

\begin{enumerate}
\item	Se tiene una computadora MIPS32 con un reloj de 1.5 GHz, una latencia de memoria de 80 ns y un ancho de banda de 400MHz. El CPI de ejecución de esta CPU es de 1. En esta computadora se ejecuta la función \texttt{memcpy (void *dst, void *src, size\_t size)} con los argumentos \texttt{0xbebacafe}, \texttt{0xbebac0c0}, y \texttt{32}. La función se encuentra en la dirección \texttt{0xcafe0000}

\begin{verbatim}
        .globl  memcpy
        .ent    memcpy
memcpy:
        addu    $a2,$a2,-1
        li      $v0,-1                  # 0xffffffffffffffff
        beq     $a2,$v0,$L7
        move    $v1,$a0
        li      $a3,-1                  # 0xffffffffffffffff
$L5:
        lbu     $v0,0($a1)
        sb      $v0,0($v1)
        addu    $a1,$a1,1
        addu    $a2,$a2,-1
        addu    $v1,$v1,1
        bne     $a2,$a3,$L5

$L7:
        move    $v0,$a0
        j       $ra                     # return
\end{verbatim}
\begin{enumerate}
\item ¿Respeta este programa la ABI dada en clase? ¿Por qué?
\item ¿Es compatible este programa con código que esté compilado respetando la ABI dada en clase? ¿Por qué?
\item ¿Cuál es el CPI efectivo de esta ejecución de la función? ¿Cuánto dura la ejecución?
\item ¿Qué proporción de ese CPI representan los ciclos de stall?
\end{enumerate}
\item En la computadora anterior se agrega una memoria caché split de mapeo directo de 128 KB en total, WB/WA, con bloques de 8 bytes. La caché de instrucciones y la de datos tienen el mismo tamaño. 
\begin{enumerate}
\item ¿A qué conjuntos mapean los vectores y la función?
\item ¿Cuál es el CPI efectivo de esta ejecución de la función? ¿Cuánto tarda?
\item ¿Cuál es el speedup para el tiempo promedio de acceso a memoria contra no usar caché?
\item ¿Se verifica la Ley de Amdahl?
\end{enumerate}


\item Se dispone de una computadora con 16MiB de memoria. El soporte para memoria virtual se implementa mediante una Tabla de páginas invertida, asistida por una \texttt{HAT} y tamaño de página de  4MiB. El proceso de traducción de memoria solo considera los 24 bits menos significativos de la dirección.

\begin{enumerate}
    \item Indique cómo se utiliza la ubicación virtual para obtener la dirección física.
    \item ¿Cuántas entradas tiene la Tabla de páginas? Describa el contenido de una entrada.
    \item Si se dispone de la siguiente función de hash \texttt{h = (va $>>$ OFFSET\_BITS) \& 0x3}. ¿Cuál es la \underline{máxima} cantidad de accesos posibles para traducir una dirección? Describa como se puede producir un fallo de página en este modelo.
    \item Se dispone de un TLB asociativo de 4 entradas. Traduzca las siguientes direcciones virtuales.
    
    \begin{itemize}
        \item \texttt{0x00078A}
        \item \texttt{0x0501AB}
        \item \texttt{0x0FF001}
    \end{itemize}
    
    Contenido del TLB:
    \begin{center}
    \begin{tabular}{|c|c|}
    \hline
        VPN & PPN   \\
        \hline
        0x0CA & 0x00E \\
        0x000 & 0x0A0 \\
        0x0FF & 0x055 \\
        0x0DE & 0x042 \\
        \hline
    \end{tabular}
    \end{center}
\end{enumerate}\end{enumerate}


\end{document}
