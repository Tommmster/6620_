\documentclass{article}

\title{Trabajo práctico 1: Conjunto de instrucciones MIPS}
\begin{document}

\date{}
\maketitle
\section{Objetivos}
Este trabajo práctico tiene como objetivo familiarizarse con el conjunto de instrucciones
MIPS 32, así como identificar posibilidades de mejora sobre la solución obtenida.

\section{Recursos}
En la clase del jueves 4/4 presentamos las herramientas necesarias para poder compilar y ejecutar
programas escritos en assembly MIPS 32 y el ABI requerido para poder ejecutar llamadas a 
función desde código C utilizando el compilador \texttt{GCC}

\section{Descripción}
El trabajo práctico consiste en implementar un problema denominado \textbf{"La hormiga artista"}. El mismo plantea un recorrido
sobre un espacio de dos dimensiones, donde el mismo está definido por reglas predefinidas.

Se dispone de un area bidimensional dividida en celdas cuyo tamaño es configurable, donde cada celda está pintada de un color 
\footnote{El color inicial de todo el espacio es el primero de la paleta especificada}. En la posición central se encuentra una
hormiga, orientada hacia el norte, que dispone de una paleta de colores. Al moverse a una celda, la hormiga cambia su orientación 
rotando hacia la izquierda o derecha, pinta la celda utilizando el siguiente color de la paleta y se mueve a la celda que tiene adelante. 
Esta secuencia de acciones se repite un número de veces predeterminado.

Finalmente se escribe el estado de la grilla por la salida estandar, en formato PPM.


\subsection{Argumentos}

El programa debe tomar las siguientes opciones
\begin{itemize}
\item Las dimensiones de la grilla
\item La paleta de colores
\item El conjunto de reglas para realizar las rotaciones
\item La cantidad de movimientos a realizar
\end{itemize}

Los colores disponibles son
\begin{itemize}
\item Rojo (R)
\item Azul (B)
\item Verde (G)
\item Amarillo (Y)
\item Blanco (W)
\item Negro (N)
\end{itemize}

\subsection{Implementación}
El punto de partida de este trabajo práctico es un programa escrito en código C, en el cual se encuentran resueltos
aspectos como la evaluación de parámetros o la impresión final. Sin embargo, el movimiento de la hormiga a través del
espacio bidimensional debe programarse y tiene la siguiente interfaz:

\begin{verbatim}
        void*
        paint (void *ant, void *grid, void *palette, uint32_t iterations)
\end{verbatim}

Si bien no se fuerza una implementación de las distintas estructuras, se recomienda las siguientes:

\begin{verbatim}
        typedef struct {
            unsigned int x, y; /* Cell coordinates */
            unsigned int o; /* orientation */
        } ant_t;

        typedef struct {
            unsigned int []colours;
            unsigned int current;
        } palette_t
\end{verbatim}

Mientras que la siguiente es la implementación más simple del espacio bidimensional
\begin{verbatim}
        typedef struct {
            unsigned int [][]colour; /* cell state */
            unsigned int w, h; /* grid width & height */
        } square_grid_t;
\end{verbatim}

\subsection{Código MIPS}
Al igual que al programar en un lenguaje de alto nivel, se recomienda modularizar el código en funciones
cortas y con un único propósito. Adicionalmente, si bien es posible llamar funciones C desde código assembly,
es preferible portar las funciones llamadas a MIPS 32, respetando el ABI de la cátedra.\footnote{Sin embargo es posible escribir funciones en C que ayuden
al depurar el programa}.

Con el fin de obtener un programa \textit{correcto}, se recomienda escribir una versión inicial que no contemple
optimizaciones prematuras. Por ejemplo, utilizando el \textit{stack} para variables locales y realizando una traducción
de código C a \textit{assembly}. Posteriores versiones pueden implementar mejoras como \underline{tablas de saltos},
variables locales en registros o variaciones al ABI.

\section{Ejemplos}

    \begin{verbatim}
    $ tp1 --grid 3x3 --palette R|G|B --rules L|R|L --times 3
    P3
    3 3
    255

    255 0   0   255 0   0   255 0   0
    0   0   255 0   255 0   255 0   0 
    255 0   0   255 0   0   255 0   0
    \end{verbatim}

    \begin{verbatim}
    $ tp1 --grid 3x3 --palette Y|N|B --rules L|R|L --times 2
    P3
    3 3
    255

    255 255 0   255 255 0   255 255 0
    0   0   255 0   0   0   255 255 0
    255 255 0   255 255 0   255 255 0
    \end{verbatim}

    \begin{verbatim}
    $ tp1 --grid 2x2 --palette N|W --rules R|L --times 1
    P3
    2 2
    255

    0   0   0   0   0   0 
    0   0   0   255 255 255
    \end{verbatim}

\section{Extras}
\begin{itemize}
\item Mediciones de tiempo
\item Impacto sobre el \textit{cache}
\end{itemize}

\section{Condiciones de entrega}
Las fechas de entrega para el trabajo práctico son el Jueves 25/4 y el jueves 9/5. La entrega debe incluir un informe
describiendo la resolución del trabajo práctico, que incluya:

\begin{itemize}
\item Carátula especificando los datos y contacto de los integrantes del grupo (dirección de correo electrónico, \textit{handle} de slack)
\item Instrucciones necesarias para compilar y ejecutar el programa
\item Decisiones relevantes sobre la implementación y resolución
\item Conclusiónes con \underline{fundamentos reales}
\item Diagramas ilustrando la estructura del \textit{stack} de cada función relevante
\item Código fuente
\item Este enunciado
\end{itemize}


\end{document}
