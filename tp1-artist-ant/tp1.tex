\documentclass{article}

%\usepackage[spanish]{babel}
%\usepackage[ansinew]{inputenc}
%% graficos
%\usepackage[pdftex]{graphicx}

%\usepackage{graphicx}
%\usepackage[ansinew]{inputenc}
%\usepackage[spanish]{babel}
%\usepackage{multirow}
%\usepackage{fancyhdr}
%\usepackage{url}
%\usepackage{amsmath}


\author{}
\title{Trabajo práctico 1: Conjunto de instrucciones MIPS}

\begin{document}
\date{}
\maketitle

\section{Introducción}
Este trabajo práctico tiene como objetivo familiarizarse con el conjunto de instrucciones
MIPS 32, así como identificar posibilidades de mejora sobre la solución obtenida.

En la clase del jueves 4/4 presentamos las herramientas necesarias para poder compilar y ejecutar
programas escritos en assembly MIPS 32 y el ABI requerido para poder ejecutar llamadas a 
función desde código C utilizando el compilador \texttt{GCC}

\section{Descripción}
El trabajo práctico consiste en implementar un problema denominado \textbf{"La hormiga artista"}. El mismo plantea un recorrido
sobre un espacio de dos dimensiones, donde el mismo está definido por reglas predefinidas.

Se dispone de un area bidimensional dividida en celdas cuyo tamaño es configurable, donde cada celda está pintada de un color 
\footnote{El color inicial de todo el espacio es el primero de la paleta especificada}. En la posición central se encuentra una
hormiga, orientada hacia el norte\footnote{En el formato de salida, hacia el encabezado del archivo}, que dispone de una paleta de colores. 
Al enccontrarse sobre una celda, la hormiga cambia su orientación rotando hacia la izquierda (\texttt{L}) o derecha (\texttt{R}) según el 
color sobre el que se encuentre, pinta la celda utilizando el siguiente color de la paleta \footnote{El primer color a utilizar es el segundo 
color de la paleta} y se mueve a la celda que tiene adelante. Esta secuencia de acciones se repite un número de veces predeterminado.
Finalmente se escribe el estado de la grilla por la salida estandar, en formato PPM.

La paleta de colores contiene una combinación de los siguientes colores: 
Rojo (\texttt{R}), Azul (\texttt{B}), Verde (\texttt{G}), Amarillo (\texttt{Y}), Blanco (\texttt{W}) y Negro (\texttt{N}).


\section{El programa}

El programa debe tomar las siguientes opciones
\begin{itemize}
\item Las dimensiones de la grilla
\item La paleta de colores
\item El conjunto de reglas para realizar las rotaciones
\item La cantidad de movimientos a realizar
\end{itemize}


\subsection{Implementación}
Como punto de partida se provee un esqueleto escrito en C en el cual se encuentran resueltos aspectos como la evaluación de parámetros, 
algunas estructuras y la impresión final en formato PPM.  Sin embargo, el movimiento de la hormiga a través del
espacio bidimensional debe programarse y tiene la siguiente interfaz:

\begin{small}
\begin{verbatim}
void* paint(void *ant, void *grid, void *palette, void *rules,  uint32_t times);
\end{verbatim}
\end{small}

Asimismo, la implementación de la paleta de colores y el conjunto de reglas debe ser realizado como parte de la entrega. Existen dos funciones
encargadas de interpretar dichos parámetros - \texttt{make\_palette(char *)}, \texttt{make\_grid(char *)} - que en su versión \textit{default}
detienen la ejecución del programa.

Dentro del archivo \texttt{ant\_constants.h} contiene constantes predeterminadas para las orientaciones, rotaciones y colores. 

Como parte de la entrega se pide escribir dos versiones de este programa. La primera, escrita íntegramente en \texttt{C}, completando
el esqueleto. La segunda versión, implementando la función \texttt{paint} en assembly MIPS.

\subsubsection{Código MIPS}
Se debe disponer de una implementación de la función \texttt{paint} en código \textit{assembly} MIPS, \underline{que respete el ABI 
de la cátedra}. Cualquier función llamada desde dicha función debe estar implementada en \textit{assembly}. Si bien es posible llamar i
funciones C desde código assembly, es preferible limitar su uso a funciones de depuración.

Con el fin de obtener un programa \textit{correcto}, se recomienda escribir una versión inicial que no contemple
optimizaciones prematuras. Por ejemplo, utilizando el \textit{stack} para variables locales y realizando una traducción
de código C a \textit{assembly}. Posteriores versiones pueden implementar mejoras como \underline{tablas de saltos},
variables locales en registros o variaciones al ABI.

\section{Ejemplos}

    \begin{verbatim}
    $ tp1 --grid 3x3 --palette "R|G|B" --rules "L|R|L" --times 3
    P3
    3 3
    255

    255 0   0   255 0   0   255 0   0
    0   0   255 0   255 0   255 0   0 
    255 0   0   255 0   0   255 0   0
    \end{verbatim}

    \begin{verbatim}
    $ tp1 --grid 3x3 --palette Y|N|B --rules L|R|L --times 2
    P3
    3 3
    255

    255 255 0   255 255 0   255 255 0
    0   0   255 0   0   0   255 255 0
    255 255 0   255 255 0   255 255 0
    \end{verbatim}

    \begin{verbatim}
    $ tp1 --grid 2x2 --palette N|W --rules R|L --times 1
    P3
    2 2
    255

    0   0   0   0   0   0 
    0   0   0   255 255 255
    \end{verbatim}

\section{Extras}
\begin{itemize}
\item Mediciones de tiempo
\item Impacto sobre el \textit{cache}
\end{itemize}

\section{Recursos}

\begin{itemize}
\item ABI de la cátedra
\item Código inicial
\item HP - The hardware software interface
\item HP - CAAQA
\end{itemize}

\section{Condiciones de entrega}

Las fechas de entrega para el trabajo práctico son el Jueves 25/4 y el jueves 9/5. La entrega debe incluir un informe
describiendo la resolución del trabajo práctico, que incluya:

\begin{itemize}
\item Carátula especificando los datos y contacto de los integrantes del grupo (dirección de correo electrónico, \textit{handle} de slack)
\item Instrucciones necesarias para compilar y ejecutar el programa
\item Decisiones relevantes sobre la implementación y resolución
\item Conclusiones con \underline{fundamentos reales}
\item Diagramas ilustrando la estructura del \textit{stack} de cada función relevante
\item Casos de prueba documentados
\item Código fuente
\item Este enunciado
\end{itemize}


\end{document}
