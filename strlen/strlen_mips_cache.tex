\documentclass[a4paper]{article}

\usepackage{listings}
\usepackage{color}

\usepackage[spanish]{babel}
\selectlanguage{spanish}
\usepackage[utf8]{inputenc}
\usepackage{enumerate}
\usepackage{fancyhdr}

\pagestyle{fancy}

\fancyhead{}
\fancyfoot{}

\fancyfoot[R]{\thepage}
\fancyhead[L]{6620 - Organización de Computadoras.}
\title{Impacto de la función \textit{strlen} en la jerarquía de memoria}
\date{}
\begin{document}
\maketitle


\section{Disposición de datos en el \textit{caché}}
Se dispone de una computadora MIPS32 que dispone de un \textit{caché} 2WSA, $WT/\neg WA$,capacidad total 128 bytes,
bloques de 8 bytes y política de reemplazo LRU.

\begin{enumerate}

\item Indicar que bits de las direcciones emitidas por el procesador se corresponden con los campos \textit{offset}, \textit{index} y \textit{tag}.

\item Indicar el contenido final del \textit{cache}, así como la tasa de desaciertos (\texttt{MR}) al ejecutar la 
función \texttt{strlen} con el siguiente argumento:

\begin{lstlisting}
const char *s = lplusd; /* point to a 26 bytes long C-string */
strlen(s);
\end{lstlisting}
 
 Se puede suponer que el registro \texttt{\$sp} contiene el valor \texttt{0xFFFF0000} y que la dirección representada por \texttt{s} está alineada con el \textit{cache}\footnote{El dato apuntado por \texttt{s} se corresponde con el primer byte del primer bloque.}
\begin{enumerate}[a.]
  
  \item A partir de la implementación provista en la sección \ref{sec:cod_c} y considerando \textbf{únicamente} 
	los accesos al arreglo de caracteres.
 
  \item A partir de la implementación provista en la sección \ref{sec:cod_asm} y considerando \textbf{únicamente} 
	los accesos al \textit{stack} de la función \texttt{strlen}.
  
  \item Considerando \textbf{únicamente} el acceso a memoria para realizar el \textit{fetch} de instrucciones.


  \item Repetir los dos puntos anteriores, utilizando la implementación provista en la sección \ref{sec:cod_asm_r}
\end{enumerate}

\item ¿Es posible realizar alguna afirmación sobre el tiempo de ejecución de las distintas variantes en esta computadora?

\end{enumerate}

\pagebreak

\appendix
\section{Código fuente}
\subsection{Implementación en C}
\label{sec:cod_c}
\lstinputlisting[language=C,basicstyle=\footnotesize]{code/strlen.c}

\pagebreak
\subsection{Implementación en assembly Mips32}
\subsubsection{Utilizando el \textit{stack} para almacenar variables locales}
\label{sec:cod_asm}
\lstinputlisting[language=C,basicstyle=\footnotesize]{code/strlen.S}

\pagebreak
\subsubsection{Utilizando registros para almacenar variables locales}
\label{sec:cod_asm_r}
\lstinputlisting[language=C,basicstyle=\footnotesize]{code/strlen_2.S}

\end{document}
