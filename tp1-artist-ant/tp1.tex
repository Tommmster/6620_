\documentclass{article}


\title{Trabajo práctico 1: Conjunto de instrucciones MIPS}
\begin{document}

\date{}
\maketitle
\section{Objetivos}
Este trabajo práctico tiene como objetivo familiarizarse con el conjunto de instrucciones
MIPS 32, así como identificar posibilidades de mejora sobre la solución obtenida.

\subsection{Recursos}
En la clase del jueves 4/4 presentamos las herramientas necesarias para poder compilar y ejecutar
programas escritos en assembly MIPS 32 y el ABI requerido para poder ejecutar llamadas a 
función desde código C utilizando el compilador \texttt{GCC}


\section{Descripción}
El trabajo práctico consiste en implementar un problema denominado "La hormiga artista". El mismo plantea un recorrido
sobre un espacio de dos dimensiones, donde el mismo está definido por reglas predefinidas.

Se dispone de un area bidimensional dividida en celdas cuyo tamaño es configurable, donde cada celda está pintada de un color 
\footnote{El color inicial de todo el espacio es el primero de la paleta especificada}. Dentro de esta area se encuentra una
hormiga que dispone de una paleta de colores. Al moverse a una celda, la hormiga cambia su orientación rotando hacia la izquierda o 
derecha, pinta la celda utilizando el siguiente color de la paleta y se mueve a la celda que tiene adelante. Esta secuencia de 
acciones se repite un número de veces.

\subsection{Argumentos}

El programa debe tomar las siguientes opciones
\begin{itemize}
\item Las dimensiones de la grilla
\item La paleta de colores
\item El conjunto de reglas para realizar las rotaciones
\item La cantidad de movimientos a realizar
\end{itemize}

Una vez agotadas las iteraciones, el programa imprime el estado de la grilla en \textit{stdout}

\subsection{Implementación}

\begin{verbatim}
        void*
        paint (void *ant, void *grid, void *palette, uint32_t iterations)
\end{verbatim}

\section{Código MIPS}

\section{Ejemplos}

\section{Entrega}

Las fechas de entrega para el trabajo práctico son el Jueves 25/4 y el jueves 9/5. La entrega debe incluir un informe
describiendo la resolución del trabajo práctico, que incluya:

\begin{itemize}
\item Carátula especificando los datos y contacto de los integrantes del grupo (dirección de correo electrónico, \textit{handle} de slack)
\item Instrucciones necesarias para compilar y ejecutar el programa
\item Decisiones relevantes sobre la implementación y resolución
\item Conclusiónes con fundamentos reales
\item Código fuente
\item Este enunciado
\end{itemize}

\end{document}
