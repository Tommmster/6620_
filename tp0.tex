\documentclass[9pt,a4paper]{article}

\usepackage[spanish]{babel}
\selectlanguage{spanish}
\usepackage[utf8]{inputenc}
\usepackage{geometry}
\usepackage{url}
\usepackage{amsmath}
\usepackage{graphicx}
\usepackage{listings}
\usepackage{fancyhdr}

\newcommand{\stdout}{\texttt{stdout}}
\newcommand{\stdin}{\texttt{stdin}}
\newcommand{\stderr}{\texttt{stderr}}
\newcommand{\gprof}{\texttt{gprof}}
\newcommand{\nl}{\texttt{nl}}
\newcommand{\unixtime}{\texttt{time}}

\pagestyle{fancy}

\title{6620: Organización de computadoras.\\ Trabajo práctico 0: Infraestructura básica}
\begin{document}

\maketitle
\section{Objetivos}

\section{Recursos}

\section{Introducción}
El trabajo práctico consiste en implementar y optimizar una versión minimalista
del comando \nl\cite{NL} de Unix. El mismo recibe texto e imprime por \stdout{} las 
lineas numeradas.

Una vez implementado, se debe utilizar
\texttt{gprof}\cite{GPROF} para buscar las secciones del programa que consumen
mas tiempo de ejecución

\section{El programa}
El programa debe leer el contenido de uno o mas archivos y por cada linea, imprimir el número de linea. En caso de 
que no se indique ningún archivo, entonces debe leerse de \stdin.

Esta versión de \nl{} solo debe implementar las siguientes opciones:
\begin{itemize}
\item \texttt{-s}, \texttt{--number-separator}. Indica el texto separador entre número de linea y la linea.
\item \texttt{-v}, \texttt{--starting-line-number}. Indica el número de la primer linea.
\item \texttt{-i}, \texttt{--line-increment}. Indica el incremento entre lineas consecutivas.
\item \texttt{-t}, \texttt{--non-empty}. Si está presente, solo se deben numerar las lineas no vacias. Caso contrario,
también deben numerar las lineas vacias 
\item \texttt{-l}, \texttt{--join-blank-lines}. Indica la cantidad de lineas vacias a agrupar en una única linea.
\item \texttt{-h}, \texttt{--help}. Imprime el mensaje de ayuda.
\end{itemize}

Los resultados deben ser impresos en \stdout{}, mientras que los mensajes de error u otras cuestiones deben imprimirse por \stderr{}.

\subsection{Mediciones}
Una vez que el programa haya sido implementado y verificado, el programa debe ser optimizado. Con este objetivo en mente, debe
utilizarse \gprof{} para analizar corridas del programa y determinar cuales son las funciones aptas para ser optimizadas.

Mediante el comando \unixtime\cite{TIME} es posible medir el tiempo de ejecución del programa. Por cada versión del programa
se debe calcular el \textit{speed up} obtenido.

\section{Ejemplos}

\begin{verbatim}
$tp0 -h

\end{verbatim}

\begin{verbatim}
$tp0 -s="-->" -v 3 --line-increment=2 in.txt
3-->While Belgium is bleeding and hoping, while Poland suffers and dreams of
5-->liberation, while Serbia is waiting for redemption, there is a little
7-->country the soul of which is torn to pieces--a little country that is so
9-->remote, so remote that her ardent sighs cannot be heard.
$
\end{verbatim}

\begin{verbatim}
$cat dante.txt | tp0 -t
1-->O'er better waves to speed her rapid course
2-->The light bark of my genius lifts the sail,
3-->Well pleas'd to leave so cruel sea behind;

4-->And of that second region will I sing,
5-->In which the human spirit from sinful blot
6-->Is purg'd, and for ascent to Heaven prepares.
\end{verbatim}


\section{Entregables}
El trabajo práctico debe ser entregado en un folio A4, acompañado por un CD. Estos
deben incluir:
\begin{itemize}
\item El informe, incluyendo los objetivos del trabajo práctico, las conclusiones y todos los 
datos utilizados en su elaboración. Esto incluye corridas de \gprof, \unixtime, y mediciones relevantes
\item El código fuente en C.
\item El código assembly generado por el compilador, \textbf{solo en formato digital}.
\item Este enunciado.
\end{itemize}

\section{Fechas de entrega}
\begin{itemize}
\item Fecha de entrega
\item Fecha de vencimiento
\end{itemize}

\begin{thebibliography}{99}
\bibitem{NL}\nl{} man page, \url{http://www.manpagez.com/man/1/nl/}
\bibitem{GPROF} \gprof{} manual, \url{https://www.cs.utah.edu/dept/old/texinfo/as/gprof.html}
\bibitem{TIME} \unixtime{} man page  , \url{http://linux.die.net/man/1/time}
\end{thebibliography}
\end{document}
